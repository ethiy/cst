\documentclass{beamer}

\usepackage[utf8]{inputenc}
\usepackage[T1]{fontenc}

\usepackage[acronym]{glossaries}

\usepackage{enumitem}

\usepackage{xcolor}

\usepackage{booktabs}

\usepackage[font={small, color=IGNDarkGrey}, labelfont={bf, color=IGNGreen}]{caption}


\usetheme{ign}


\newacronym{lod}{LoD}{Level of Detail}
\newacronym{lidar}{LiDAR}{Light Detection and Ranging}
\newacronym{dsm}{DSM}{Digital Surface Model}

\title{Semantic $3D$ building model diagnostic}
\subtitle{}
\institute[LaSTIG MATIS]{Univ. Paris Est, LaSTIG MATIS, IGN, ENSG}
\date{\today}
\author[O.Ennafii]{Oussama Ennafii}

\begin{document}

    \begin{frame}[plain]
        \titlepage{}
    \end{frame}

    \section{Introduction}
        \subsection{Context}
            \begin{frame}{What is a $3D$ model?}
                \begin{itemize}[label=$\blacktriangleright$, font=\color{IGNGreen}]
                    \item<1-> $3D$ urban model $\equiv$ polyhedral surface modeling a building;
                    \item<2-> Holds higher semantic information than a $3D$ mesh while using less memory;
                \end{itemize}
                \uncover<3->{
                    \begin{figure}[H]
                        \begin{center}
                            \includegraphics[height=.2\textheight]{images/citygml_lod}
                            \caption{\label{fig::lods_citygml} \gls{lod}~\cite{kolbe2005citygml} representation as defined in the \emph{cityGML} format~\cite{ohori2016higher}.}
                        \end{center}
                    \end{figure}
                }
            \end{frame}

            \begin{frame}{What is urban reconstruction?}
                \begin{itemize}[label=$\blacktriangleright$, font=\color{IGNGreen}]
                    \item<1-> The aim is to model urban objects, on a large scale, using:
                    \begin{itemize}[label=--]
                        \item<2-> unstructured data: \gls{lidar};
                        \item<3-> structured data: stereoscopic images.
                    \end{itemize}
                    \item<3-> The model \gls{lod} depends on:
                    \begin{itemize}[label=--]
                        \item<4-> their intended use,
                        \item<5-> the input data spatial resolution.
                    \end{itemize}
                \end{itemize}
            \end{frame}

            \begin{frame}{Urban models applications}
                \begin{itemize}[label=$\blacktriangleright$, font=\color{IGNGreen}]
                    \item Urban models have a wide application range~\cite{Biljecki2015} (\textit{c.f.} Table~\ref{tab::3d_applications});
                \end{itemize}
                \begin{table}
                    \begin{center}
                        \begin{tabular}{l l l}
                            \toprule
                            Planning & Simulation & Visualization \\
                            \midrule
                            City planning & Micro climates & Architecture \\
                            Emergency intervention & Wave propagation & Cadastre \\
                            Home decoration & Run-off water & Tourism \\
                            Communication network & Military intervention & Video games \\
                            \bottomrule
                        \end{tabular}
                        \caption{\label{tab::3d_applications} Some of the main thematic applications of $3D$ urban models~\cite{Biljecki2015, Scholze2002}.}
                    \end{center}
                \end{table}
            \end{frame}
        \subsection{Motivation}
            \begin{frame}{The need for quality assessement}
                \begin{itemize}[label=$\blacktriangleright$, font=\color{IGNGreen}]
                    \item<1-> Automatic urban modeling is an active research area~\cite{Musialski2012}, but not \textcolor{IGNRed}{yet operational}~\cite{rottensteiner2014results};
                    \item<2-> Urban $3D$ model semantic diagnostic is not well studied~\cite{nguatem2017modeling}.
                \end{itemize}
                ~\\
                \begin{itemize}[label=Goal $\longrightarrow$, font=\color{purple}, leftmargin=2cm]
                    \item Detect and describe semantic errors that affects building $3D$ models.
                \end{itemize}
            \end{frame}

            \begin{frame}{Potential use}
                \begin{itemize}[label=$\blacktriangleright$, font=\color{IGNGreen}]
                    \item<1-> Change detection;
                    \item<2-> Urban models correction;
                    \item<3-> Urban reconstruction method evaluation;
                    \item<4-> Crowd reconstruction quality assessment.
                \end{itemize}
            \end{frame}
    \section{State of the art}
    \section{Error taxonomy}
    \section{Methodology}
    \section{Used Tools}
    \section{Conclusion}

    \section*{References}
        \begin{frame}[allowframebreaks]{References}
            \bibliographystyle{alpha}
            \bibliography{references.bib}
        \end{frame}
\end{document}
